% Filename: lista03.tex
% This code is part of 'Lista de Matrizes'.
% 
% Description: Lista 03.
% 
% Created: 11.03.12 08:00:00 AM
% Last Change: 25.04.12 09:12:25 AM
% 
% Author: <+AUTHOR+>, <+EMAIL+>
% Organization: <+ORGANIZATION+>
% 
% Copyright (c) <+YEAR+>, <+AUTHOR+>. All rights reserved.
% 
% This file is license under the terms 
%
\documentclass[a4paper,12pt, leqno, answers]{exam}
\usepackage[top=3cm, bottom=3cm, left=2cm, right=2cm]{geometry}
\usepackage[utf8]{inputenc}
\usepackage[brazil]{babel}
\usepackage{amsmath}
\usepackage{amsfonts}
\usepackage{hyperref}
\usepackage{tikz}

% Customiza\c{c}\~{a}o da classe exam
\firstpageheader{MT402}{Solu\c{c}\~{a}o da Lista 3}{1º semestre de 2012}
\firstpageheadrule
\footer{Dispon\'{i}vel em \\\url{https://github.com/r-gaia-cs/solucoes_lista_matrizes}
}{}{Reportar erros para \\\href{mailto:r.gaia.cs@gmail.com}{r.gaia.cs@gmail.com}
}
\footrule 
\pagestyle{foot}
\renewcommand{\solutiontitle}{\noindent\textbf{Solu\c{c}\~{a}o:}\enspace}
\SolutionEmphasis{\itshape}
\unframedsolutions
\pointname{}

% Customiza\c{c}\~{a}o do pacote amsmath
\allowdisplaybreaks[4]

%Novos ambientes
\newenvironment{fwsolution}{\begin{EnvFullwidth}\begin{TheSolution}}{\end{TheSolution}\end{EnvFullwidth}}

% Novos comandos
%\newcommand{\mdot}{\text{\LARGE $\boldsymbol{\cdot}$}}
\newcommand{\mdot}{\bullet}
\newcommand{\EI}{\text{Im}}
\newcommand{\EN}{\text{N}}
\newcommand{\posto}{\text{posto}}

\begin{document}
\thispagestyle{headandfoot}
\begin{questions}
    \section*{Matrizes com Estruturas de Blocos}
    \question[Ver exemplo 3.7.3, p\'{a}gina 119, do Meyer\nocite{Meyer:2000:matrix}] Considere as matrizes n\~{a}o singulares: $A: r \times s$ e $B: s \times r$.
    \begin{parts}
        \part Demonstre (sem usar multiplica\c{c}\~{a}o direta), que se
        \[
        M = \begin{bmatrix}
            A & 0 \\
            0 & B
        \end{bmatrix}
        \]
        ent\~{a}o,
        \[
        M^{-1} = \begin{bmatrix}
            A^{-1} & 0 \\
            0 & B^{-1}
        \end{bmatrix}.
        \]
        \begin{solution}
            \begin{align*}
                \left[ \begin{array}[]{cc|cc}
                    A & 0 & I & 0 \\
                    0 & B & 0 & I
                \end{array} \right] &\equiv \left[ \begin{array}[]{cc|cc}
                    I & 0 & A^{-1} & 0 \\
                    0 & B & 0 & I
                \end{array} \right] \\
                &\equiv \left[ \begin{array}[]{cc|cc}
                    I & 0 & A^{-1} & 0 \\
                    0 & I & 0 & B^{-1}
                \end{array} \right].
            \end{align*}
        \end{solution}

        \part Se
        \[
        M = \begin{bmatrix}
            A & C \\
            0 & B
        \end{bmatrix}
        \]
        ent\~{a}o,
        \[
        M^{-1} = \begin{bmatrix}
            A^{-1} & - A^{-1} C B^{-1} \\
            0 & B^{-1}
        \end{bmatrix}.
        \]
        \begin{solution}
            \begin{align*}
                \left[ \begin{array}[]{cc|cc}
                    A & C & I & 0 \\
                    0 & B & 0 & I
                \end{array} \right] &\equiv \left[ \begin{array}[]{cc|cc}
                    I & A^{-1} C & A^{-1} & 0 \\
                    0 & B & 0 & I
                \end{array} \right] \\
                &\equiv \left[ \begin{array}[]{cc|cc}
                    I & A^{-1} C & A^{-1} & 0 \\
                    0 & I & 0 & B^{-1}
                \end{array} \right] \\
                &\equiv \left[ \begin{array}[]{cc|cc}
                    I & 0 & A^{-1} & -A^{-1} C B^{-1} \\
                    0 & I & 0 & B^{-1}
                \end{array} \right].
            \end{align*}
        \end{solution}

        \part Se
        \[
        M = \begin{bmatrix}
            A & C \\
            D & B
        \end{bmatrix},
        \]
        $C: r \times s$ e $D: s \times r$, ent\~{a}o
        \[
        M ^{-1} = \begin{bmatrix}
            A^{-1} + A^{-1} C S^{-1} D A^{-1} & -A^{-1} C S^{-1} \\
            -S^{-1} D A^{-1} & S^{-1}
        \end{bmatrix},
        \]
        onde $S = B - D A^{-1} C$ \'{e} o complemento de Schur de $A$. Observe que a condi\c{c}\~{a}o que $S$ tem que ser invers\'{i}vel surge no desenvolvimento do c\'{a}lculo da inversa de $M$.
        \begin{solution}
            \begin{align*}
                \left[ \begin{array}[]{cc|cc}
                    A & C & I & 0 \\
                    D & B & 0 & I
                \end{array} \right] &\equiv \left[ \begin{array}[]{cc|cc}
                    I & A^{-1} C & A^{-1} & 0 \\
                    D & B & 0 & I
                \end{array} \right] \\
                &\equiv \left[ \begin{array}[]{cc|cc}
                    I & A^{-1} C & A^{-1} & 0 \\
                    0 & B - D A^{-1} C & - D A^{-1} & I
                \end{array} \right] \\
                &\equiv \left[ \begin{array}[]{cc|cc}
                    I & A^{-1} C & A^{-1} & 0 \\
                    0 & S & - D A^{-1} & I
                \end{array} \right] \\
                &\equiv \left[ \begin{array}[]{cc|cc}
                    I & A^{-1} C & A^{-1} & 0 \\
                    0 & I & S^{-1} (- D A^{-1}) & S^{-1}
                \end{array} \right] \\
                &\equiv \left[ \begin{array}[]{cc|cc}
                    I & 0 & A^{-1} + (A^{-1} C) S^{-1} (D A^{-1}) & -A^{-1} C S^{-1} \\
                    0 & I & S^{-1} (- D A^{-1}) & S^{-1}
                \end{array} \right].
            \end{align*}
        \end{solution}

        \part Considere $M: (r + s) \times (r + s)$ dada por
        \[
        M = \begin{bmatrix}
            A & C \\
            D & B
        \end{bmatrix},
        \]
        com $A: r \times r$ e $B: s \times s$. Verifique que esta matriz pode ser fatorada na forma $M = W Y$ onde
        \[
        W = \begin{bmatrix}
            I_r & 0 \\
            D A^{-1} & I_s
        \end{bmatrix} \text{ e } Y = \begin{bmatrix}
            A & C \\
            0 & S
        \end{bmatrix}.
        \]
        \begin{solution}
            Para a verifica\c{c}\~{a}o utilizaremos a multiplica\c{c}\~{a}o direta:
            \begin{align*}
                M = W Y &= \begin{bmatrix}
                    I_r & 0 \\
                    D A^{-1} & I_s
                \end{bmatrix} \begin{bmatrix}
                    A & C \\
                    0 & S
                \end{bmatrix} \\
                &= \begin{bmatrix}
                    A & C \\
                    D & D A^{-1} C + S
                \end{bmatrix} \\
                &= \begin{bmatrix}
                    A & C \\
                    D & D A^{-1} C + \left( B - D A^{-1} C \right)
                \end{bmatrix} \\
                &= \begin{bmatrix}
                    A & C \\
                    D & B
                \end{bmatrix}.
            \end{align*}
        \end{solution}

        \part[Ver equa\c{c}\~{a}o (6.1.16), p\'{a}gina 467, do Meyer\nocite{Meyer:2000:matrix}] Considerando a matriz $M$ do item anterior, demonstre que $\det (M) = \det (A) \det (S)$, onde $S = B - D A^{-1} C$ \'{e} o complemento de Schur de $A$.
        \begin{solution}
            TODO
        \end{solution}

        \part Repita os dois itens anteriores trabalhando com a forma fatorada de $M$, $M = W Y$, na qual um dos fatores depende do complemento de Schur de $B$, $T = A - C B^{-1} D$, obtendo assim o resultado:
        \[
        \det (M) = \det (B) \det (T) = \det (A) \det (S)
        \]
        \begin{solution}
            Para a verifica\c{c}\~{a}o utilizaremos a multiplica\c{c}\~{a}o direta:
            \begin{align*}
                M = W Y &= \begin{bmatrix}
                    I_r & C B^{-1} \\
                    0 & I_s
                \end{bmatrix} \begin{bmatrix}
                    T & 0 \\
                    D & B
                \end{bmatrix} \\
                &= \begin{bmatrix}
                    T + C B^{-1} C & C \\
                    D & B
                \end{bmatrix} \\
                &= \begin{bmatrix}
                    A - C B^{-1} D + C B^{-1} C & C \\
                    D & B
                \end{bmatrix} \\
                &= \begin{bmatrix}
                    A & C \\
                    D & B
                \end{bmatrix}.
            \end{align*}
            
            TODO: demonstrar que $\det (M) = \det (B) \det (T)$.
        \end{solution}
    \end{parts}

    \section*{Subespa\c{c}os Fundamentais de A}
    Os quatro espa\c{c}os fundamentais associados \`{a} matriz $A$ s\~{a}o:
    \begin{itemize}
        \item Espa\c{c}o coluna de $A$ ou espa\c{c}o imagem de $A$: $\EI (A) = \left\{ y \in \mathbb{R}^m \mid y = A x, x \in \mathbb{R}^n \right\}$,
        \item N\'{u}cleo de $A$ ou espa\c{c}o nulo de $A$: $\EN (A) = \left\{ x \in \mathbb{R}^n \mid A x = 0 \right\}$,
        \item Espa\c{c}o linha de $A$ ou espa\c{c}o coluna de $A^t$: $\EI (A^t) = \left\{ x \in \mathbb{R}^n \mid x = A^t y , y \in \mathbb{R}^m \right\}$,
        \item N\'{u}cleo de $A^t$ ou espa\c{c}o nulo de $A^t$: $\EN (A^t) = \left\{ y \in \mathbb{R}^m \mid A^t y = 0 \right\}$.
    \end{itemize}

    \question Demonstre que o sistema linear $A x = b$ admite solu\c{c}\~{a}o se e somente se $\posto ([A \ b]) = \posto (A)$.
    \begin{solution}
        Primeiro vamos demonstrar que se o sistema linear $A x = b$ admite solu\c{c}\~{a}o ent\~{a}o $\posto ([A \ b]) = \posto (A)$.

        Se o sistema linear $A x = b$ admite solu\c{c}\~{a}o existe $x^*$ tal que $b = \sum_j A_{\mdot j} x^*_j$, i.e., $b$ \'{e} uma combina\c{c}\~{a}o linear de $A$. Portanto, $\posto ([A \ b]) = \posto (A)$.

        Agora vamos demonstrar que o $\posto ([A \ b]) = \posto (A)$ ent\~{a}o o sistema linear $A x = b$ admite solu\c{c}\~{a}o.

        Se $\posto ([A \ b]) = \posto (A)$ o conjunto das colunas de $[A \ b]$ \'{e} linearmente dependente, i.e., existe $(\alpha, \beta)$ tal que
        \begin{align*}
            \left( \sum_j \alpha_j A_{\mdot j} + \beta b \right) &= 0 \\
            \sum_j - \frac{\alpha_j}{\beta} A_{\mdot j} &= b.
        \end{align*}
        Logo, o sistema linear $A x = b$ admite solu\c{c}\~{a}o.
    \end{solution}

    \question Considere $x^*$ uma solu\c{c}\~{a}o de $A x = b$ e $S$ o conjunto das solu\c{c}\~{o}es de $A x = b$. Demonstre que $S = \left\{ x^* + z, \forall z \in \EN (A) \right\}$.
    \begin{solution}
        Seja $\bar{x} \in \left\{ x^* + z, \forall z \in \EN (A) \right\}$, ent\~{a}o
        \begin{align*}
            A \bar{x} &= A \left( x^* + z \right) && \bar{x} \in \left\{ x^* + z, \forall z \in \EN (A) \right\} \\
            &= A x^* + A z \\
            &= b + 0 && A x^* = b \text{ e } z \in \EN (A) \\
            &= b.
        \end{align*}
    \end{solution}

    \question Demonstre que a solu\c{c}\~{a}o do sistema linear \'{e} \'{u}nica se e somente se $\EN (A) = \left\{ 0 \right\}$.
    \begin{solution}
        Primeiro vamos demonstrar que se a solu\c{c}\~{a}o do sistema linear \'{e} \'{u}nica ent\~{a}o $\EN (A) = \left\{ 0 \right\}$.

        Seja $x^*$ a solu\c{c}\~{a}o \'{u}nica do sistema linear $A x = b$ e $x = x^* + y$ qualquer ponto do dom\'{i}nio. Ent\~{a}o
        \begin{align*}
            A x &= A \left( x^* + y \right) \\
            &= A x^* + A y \\
            &= b + A y.
        \end{align*}
        Logo, se existir $y \neq 0$ tal que $A y = 0$ o sistema linear teria mais de uma solu\c{c}\~{a}o. Como o sistema linear admite uma \'{u}nica solu\c{c}\~{a}o concluimos que $\EN (A) = \left\{ 0 \right\}$.

        Agora vamos demonstrar que se $\EN (A) = \left\{ 0 \right\}$ a solu\c{c}\~{a}o do sistema linear \'{e} \'{u}nica.

        Como demonstrado anteriormente, o conjunto das solu\c{c}\~{o}es do sistema linear $A x = b$, denotado por $S$, \'{e} dado por $S = \left\{ x^* + z, \forall z \in \EN (A) \right\}$, onde $x^*$ \'{e} uma solu\c{c}\~{a}o do sistema linear. Se $\EN (A) = \left\{ 0 \right\}$ notamos que $S = \left\{ x^* \right\}$ e portanto a solu\c{c}\~{a}o dos sistema linear \'{e} \'{u}nica.
    \end{solution}

    \question Para todo $b \in \mathbb{R}^m$, o sistema linear $A x = b$, $A : m \times n$, admite uma \'{u}nica solu\c{c}\~{a}o se e somente se $m = n$ e $\posto (A) = n$.
    \begin{solution}
        Primeiro vamos mostrar que se o sistema linear $A x = b$, $A: m \times n$, admite uma \'{u}nica solu\c{c}\~{a}o para todo $b \in \mathbb{R}^m$ ent\~{a}o $m = n$ e $\posto (A) = n$.

        Como o sistema linear admite solu\c{c}\~{a}o para todo $b \in \mathbb{R}^m$ temos que $A$ possue exatamente $m$ linhas ($m = n$) e no m\'{i}nimo $m$ colunas, sendo $m$ colunas linearmente independentes ($\posto (A) = n$). E como o sistema linear admite uma \'{u}nica solu\c{c}\~{a}o ent\~{a}o $A$ possue e$b$ \'{e} uma combina\c{c}\~{a}o linear das colunas de $A$ e as colunas de $A$ s\~{a}o linearmente independentes. 

        Agora vamos mostrar que se $m = n$ e $\posto (A) = n$ ent\~{a}o o sistema linear $A x = b$, $A: m \times n$ admite uma \'{u}nica solu\c{c}\~{a}o.
        
    \end{solution}

    \question Demonstre (sem usar o Teorema do N\'{u}cleo e Imagem) que $\EN (A) = \left\{ 0 \right\}$ se e somente se $A$ tem as $n$ colunas linearmente independentes.
    \begin{solution}
        Primeiro vamos mostrar que se $\EN (A) = \left\{ 0 \right\}$ ent\~{a}o $A$ tem as $n$ colunas linearmente independentes.

        Se $\EN (A) = \left\{ 0 \right\}$ temos, como demonstrado anteriormente, que o sistema linear $A x = b$ admite uma \'{u}nica solu\c{c}\~{a}o e isso ocorre somente se as $n$ colunas de $A$ forem linearmente independentes.

        Agora vamos mostrar que se $A$ tem as $n$ colunas linearmente independentes ent\~{a}o $\EN (A) = \left\{ 0 \right\}$.

        Se $A$ tem as $n$ colunas linearmente independentes ent\~{a}o
        \[
        \sum_j \alpha_j A_{\mdot j} = 0 \leftrightarrow \alpha_j = 0, \  \forall j.
        \]
        Logo, $\EN (A) = \left\{ 0 \right\}$.
    \end{solution}

    \question Julgue verdadeiro ou falso a afirma\c{c}\~{a}o abaixo:
    \begin{quote}
        Se $A: m \times n$ para $n > m$ ent\~{a}o $\EN (A) \neq \left\{ 0 \right\}$.
    \end{quote}
    \begin{solution}
        A afirma\c{c}\~{a}o \'{e} verdadeira pois se $A: m \times n$ para $n > m$ temos que pelo menos $\left( n - m \right)$ colunas de $A$ que s\~{a}o linearmente dependentes, i.e.,
        \[
        \exists \alpha_j \neq 0 \rightarrow \sum_j \alpha_j A_{\mdot j} = 0.
        \]
        Logo, $\EN (A) \neq \left\{ 0 \right\}$.
    \end{solution}

    \question Analisando os resultados demonstrados nos itens anteriores, fa\c{c}a um resumo colocando as condi\c{c}\~{o}es que a matriz $A: m \times n$ deve satisfazer para garatir a exist\^{e}ncia e unicidade da solu\c{c}\~{a}o de $A x = b$, para qualquer $b \in \mathbb{R}^m$.
    \begin{solution}
        Para maiores detalhes ver resumo na p\'{a}gina 70 do Meyer\nocite{Meyer:2000:matrix}.
        \begin{quote}
            O sistema linear possue solu\c{c}\~{a}o única se e somente se alguma das condi\c{c}\~{o}es abaixo for verdadeira:
            \begin{enumerate}
                \item $\text{posto}(A) = n$ que \'{e} igual ao número de vari\'{a}veis desconhecidas;
                \item n\~{a}o existem vari\'{a}veis livres;
                \item o sistema homogeneo associado possue apenas a solu\c{c}\~{a}o trivial.
            \end{enumerate}
        \end{quote}
    \end{solution}

    \question Considere o sistema linear $A x = b$ onde
    \[
    A = \begin{bmatrix}
        2 & 5 \\
        4 & 10
    \end{bmatrix} \text{ e } b = \begin{bmatrix}
        3 \\
        6
    \end{bmatrix}.
    \]
    Encontre o conjunto de solu\c{c}\~{o}es para este sistema linear e interprete geometricamente o resultado abaixo:
    \begin{quote}
        Conhecida uma solu\c{c}\~{a}o $x^*$ qualquer vetor da forma $x^* + w$ \'{e} solu\c{c}\~{a}o para o sistema linear onde $w$ \'{e} um vetor qualquer do $\EN (A)$.
    \end{quote}
    \begin{solution}
        O conjunto solu\c{c}\~{a}o do sistema linear, denotado por $S$, \'{e} dado por
        \[
        S = \left\{ (x, y) \in \mathbb{R}^2 \mid x = \left( 3 - 5 y \right) / 2 \right\}.
        \]

        \begin{center}
        \begin{tikzpicture}[scale=0.5]
            % Eixos
            \draw[color=gray!70] (0,0) grid (12,12);
            \draw[->] (-0.5,0) -- (12.5,0) node[below] {$x$};
            \draw[->] (0,-0.5) -- (0,12.5) node[left] {$y$};
            % Colunas de A
            \draw[->] (0,0) -- (2,4) node[below right] {$(2,4)$};
            \draw[->] (0,0) -- (5,10) node[below right] {$(5,10)$};
            % Vetor B
            \draw[->] (0,0) -- (3,6) node[below right] {$(3,6)$};
        \end{tikzpicture}
        \end{center}
    \end{solution}

    \question Demonstre que $\EN (A^t)$ e $\EI (A)$ s\~{a}o complementos ortogonais em $\mathbb{R}^m$. E que $\EN (A)$ e $\EI (A^t)$ s\~{a}o complementos ortogonais em $\mathbb{R}^n$.
    \begin{solution}
        Seja $y \in \EI (A)$ e $z \in \EN (A^t)$, i.e., existe $x$ tal que $A x = y$ e $A^t z = 0$. Ent\~{a}o
        \begin{align*}
            A x &= y \\
            \left( A x \right)^t &= y^t \\
            x^t A^t &= y^t \\
            x^t A^t z &= y^t z \\
            x^t 0 &= y^t z && z \in \EN (A^t) \\
            0 &= y^t z.
        \end{align*}

        Seja $y \in \EI (A^t)$ e $z \in \EN(A)$, i.e., existe $x$ tal que $A^t x = y$ e $A z = 0$. Ent\~{a}o
        \begin{align*}
            A^t x &= y \\
            \left( A^t x \right)^t &= y^t \\
            x^t A &= y^t \\
            x^t A z &= y^t z \\
            x^t 0 &= y^t z && z \in \EN (A) \\
            0 &= y^t z.
        \end{align*}
    \end{solution}

    \section*{Resultados para posto de matriz}

    \question[Exemplo 4.4.8, p\'{a}gina 206, do Meyer\nocite{Meyer:2000:matrix}] Demonstre que $\posto (A + B) \leq \posto (A) + \posto (B)$.
    \begin{solution}
        Note que
        \begin{align*}
            \EI(A + B) \subseteq \EI(A) + \EI(B)
        \end{align*}
        pois se $b \in \EI(A + B)$ ent\~{a}o existe $x$ tal que
        \begin{align*}
            b = (A + B) x = Ax + Bx \in \EI(A) + \EI(B).
        \end{align*}

        Mas se $M$ e $N$ s\~{a}o espa\c{c}os vetoriais tal que $M \subseteq N$ ent\~{a}o $\dim M \leq \dim N$. Utilizando isso com a f\'{o}rmula para a dimens\~{a}o da soma temos
        \begin{align*}
            \posto (A + B) &= \dim \EI(A + B) \\
            &\leq \dim \left( \EI(A) + \EI(B) \right) \\
            &= \dim \EI(A) + \dim \EI(B_ - \dim \left( \EI(A) \cap \EI(B) \right) \\
            &\leq \dim \EI(A) + \dim \EI(B) \\
            &= \posto (A) + \posto (B).
        \end{align*}
    \end{solution}

    \question Se $A: m \times p$ e $B: p \times n$, demonstre que:
    \begin{parts}
        \part[Equa\c{c}\~{a}o (4.5.1), p\'{a}gina 210, do Meyer\nocite{Meyer:2000:matrix}] $\posto (A B) = \posto (B) - \dim (\EN (A) \cap \EI (B))$.
        \begin{solution}
            Come\c{c}ando com uma base $S = \left\{ x_1, x_2, \ldots, x_x \right\}$ para $\EN(A) \cap \EI(B)$, notamos que $\EN(A) \cap \EI(B) \subseteq \EI(B)$. Se $\dim \EI(B) = s + t$, ent\~{a}o existe um conjunto $S_\text{ext} = \left\{ z_1, z_2, \ldots, z_t \right\}$ tal que $B = \left\{ x_1, \ldots, x_s, z_1, \ldots, z_t \right\}$ \'{e} base para $\EN(B)$. Nosso objetivo \'{e} provar que $\dim \EI(AB) = t$ e isso \'{e} feito mostrando $T = \left\{ A z_1, A z_2, \ldots, A z_t \right\}$ \'{e} base para $\EI(AB)$. $T$ gera $\EI(AB)$ porque se $b \in \EI(AB)$ ent\~{a}o $b = A B y$ para algum $y$, mas $B y \in \EI(B)$ implica em $B y = \sum_{i = 1}^s \nu_i z_i$, e portanto
            \begin{align*}
                b = A \left( \sum_{i = 1}^s \varepsilon_i x_i + \sum_{i = 1}^t \nu_i z_i \right) = \sum_{i = 1}^s \varepsilon_i A x_i + \sum_{i = 1}^t \nu_i A z_i = \sum_{i = 1}^t \nu_i A z_i.
            \end{align*}

            $T$ \'{e} linearmente independente porque se $0 = \sum_{i = 1}^t \alpha_i A z_i = A \sum_{u = 1}^t \alpha_i x_i$ ent\~{a}o $\sum_{i = 1}^t \alpha_i z_i \in \EN(A) \cap \EI(B)$. Portanto existe escalares $\beta_j$ tal que
            \begin{align*}
                \sum_{i = 1}^t \alpha_i z_i = \sum_{j = 1}^s \beta_j x_j
            \end{align*}
            e a única solu\c{c}\~{a}o para $\alpha_i$ e $\beta_j$ \'{e} a solu\c{c}\~{a}o trivial porque $B$ \'{e} um conjunto independente. Logo $T$ \'{e} base para $\EI(AB)$ e $t = \dim \EI(AB) = \posto (AB)$ e
            \begin{align*}
                \posto (B) = \dim \EI(B) = s + t = \dim \EN(A) \cap \EI(B) + \posto (AB).
            \end{align*}
        \end{solution}

        \part[Equa\c{c}\~{o}es (4.5.2) e (4.5.3), p\'{a}gina 211, do Meyer\nocite{Meyer:2000:matrix}] $\posto (A) + \posto (B) - n \leq \posto (A B) \leq \min \left\{ \posto (A), \posto (B) \right\}$.
        \begin{solution}
            Primeiro vamos mostrar que $\posto (AB) \leq \min \left\{ \posto(A), \posto(B) \right\}$.

            Escrevendo
            \begin{align*}
                \posto(AB) = \posto(B) - \dim \EN(A) \cap \EI(B) \leq \posto(B)
            \end{align*}
            e
            \begin{align*}
                \posto(AB) = \posto(AB)^t = \posto(B^t A^b) \leq \posto(A^t) = \posto(A).
            \end{align*}
            Logo,
            \begin{align*}
                \posto(A) \leq min \left\{ \posto(A), \posto(B) \right\}.
            \end{align*}

            Agora vamos mostrar que $\posto(A) + \posto(B) - n \leq \posto(AB)$.

            Notamos que $\EN(A) \cap \EI(B) \subseteq \EN(A)$ e que se $M$ e $N$ s\~{a}o espa\c{c}os vetoriais tal que $M \subseteq N$ ent\~{a}o $\dim M \leq \dim N$. Portanto,
            \begin{align*}
                \dim \EN(A) \cap \EI(B) \leq \dim \EN(A) = n - \posto (A)
            \end{align*}
            e utilizando o limitante inferior para $\posto(AB)$ encontrado anteriormente temos
            \begin{align*}
                \posto(AB) = \posto(B) = \dim \EN(A) \cap \EI(B) \geq \posto(B) + \posto(A) - n.
            \end{align*}
        \end{solution}
    \end{parts}

    \question Para matrizes $A$ e $B$ demonstre que $\EI (A B) \subseteq \EI (A)$ e $\EN (B) \subseteq \EN (A B)$.
    \begin{solution}
        Pela defini\c{c}\~{a}o de espa\c{c}o imagem temos que
        \begin{align*}
            \EI (A B) &= \left\{ y \mid A B x = y \right\}, \\
            \EI (A) &= \left\{ z \mid A w = z \right\}.
        \end{align*}
        Notamos ent\~{a}o que $\EI (A B) \subseteq \EI (A)$ pois \'{e} poss\'{i}vel que exista $w$ que n\~{a}o perten\c{c}a a imagem de $B$.

        Pela defini\c{c}\~{a}o de espa\c{c}o nulo temos que
        \begin{align*}
            \EN (A B) &= \left\{ x \mid A B x = 0 \right\}, \\
            \EN (B) &= \left\{ y \mid B y = 0 \right\}.
        \end{align*}
        Notamos ent\~{a}o que $\EN (B) \subseteq \EN (A B)$ pois para todo $y \in \EN (B)$ verifica-se $A B y = 0$ mas pode existir $x$ tal que $B x \neq 0$ mas $ A B x = 0$.
    \end{solution}

    \question Demonstre que:
    \begin{parts}
        \part[Equa\c{c}\~{a}o (4.5.4), p\'{a}gina 212, do Meyer\nocite{Meyer:2000:matrix}] $\posto (A^t A) = \posto (A) = \posto (A A^t)$.
        \begin{solution}
            Note que $\EN(A^t) \cap \EI(A) = \left\{ 0 \right\}$ pois
            \begin{align*}
                x \in \EN(A^t) \cap \EI(R) &\implies A^t x = 0, x = A y \\
                &\implies x^t x = y^t A^t x = 0 \\
                &\implies \sum x_i^2 = 0 \\
                &\implies x = 0.
            \end{align*}
            Logo,
            \begin{align*}
                \posto(A^t A) = \posto(A) - \dim \EN(A^t) \cap \EI(A) = \posto(A).
            \end{align*}
            Permutando a ordem de $A$ e $A^t$ temos
            \begin{align*}
                \posto(A A^t) = \posto(A^t) - \dim \EN(A) \cap \EI(A^t) = \posto(A^t).
            \end{align*}
        \end{solution}

        \part[Equa\c{c}\~{a}o (4.5.5), p\'{a}gina 212, do Meyer\nocite{Meyer:2000:matrix}] $\EI (A^t A) = \EI (A^t)$ e $\EI (A A^t) = \EI (A)$.
        \begin{solution}
            Note que $\EI(AB) \subseteq \EI(A)$ e portanto $\EI(A^t A) \subset \EI(A^t)$. Logo
            \begin{align*}
                \dim \EI(A^t A) = \posto(A^t A) = \posto(A) = \posto(A^t) = \dim \EN(A^t)
            \end{align*}
            e permutando a ordem de $A$ e $A^t$ temos
            \begin{align*}
                \dim \EI(A A^t) = \posto(A A^t) = \posto(A^t) = \posto(A) = \dim \EN(A).
            \end{align*}
        \end{solution}

        \part[Equa\c{c}\~{a}o (4.5.6), p\'{a}gina 212, do Meyer\nocite{Meyer:2000:matrix}] $\EN (A^t A) = \EN (A)$ e $\EN (A A^t) = \EN (A^t)$.
        \begin{solution}
            Note que $\EN(B) \subset \EN(AB)$ e portanto $\EN(A) \subset \EN(A^t A)$. Logo,
            \begin{align*}
                \dim \EN(A) = n - \posto(A) = n - \posto(A^t A) = \dim \EN(A^t A)
            \end{align*}
            e permutando a ordem de $A$ e $A^t$ temos
            \begin{align*}
                \dim \EN(A^t) = n - \posto(A^t) = n - \posto(A A^t) = \dim \EN(A A^t).
            \end{align*}
        \end{solution}
    \end{parts}

    \question[Ver resumo, p\'{a}gina 214, do Meyer\nocite{Meyer:2000:matrix}] Considere o sistema linear $A x = b$, com $A : m \times n$ e o sistema $A^t A x = A^t b$ (sistema de equa\c{c}\~{o}es normais). Verifique se as informa\c{c}\~{o}es abaixo s\~{a}o verdadeiras ou falsas.
    \begin{parts}
        \part O sistema $A^t A x = A^t b$ sempre admite solu\c{c}\~{a}o, ainda que $A x = b$ n\~{a}o tenha solu\c{c}\~{a}o.
        \begin{solution}
            Verdadeiro.
        \end{solution}

        \part Se os sitema linear $A x = b$ admite solu\c{c}\~{a}o ent\~{a}o os dois sistemas possuem o mesmo conjunto solu\c{c}\~{a}o.
        \begin{solution}
            Verdadeiro.
        \end{solution}
    \end{parts}
\end{questions}

\bibliographystyle{plain}
\bibliography{bibliography}
\end{document}


