% Filename: lista06.tex
% This code is part of 'Listas de Matrizes'
% 
% Description: Lista 06.
% 
% Created: 25.03.12 04:31:30 PM
% Last Change: 27.04.12 10:19:07 AM
% 
% Author: <+AUTHOR+>, <+EMAIL+>
% Organization: <+ORGANIZATION+>
% 
% Copyright (c) <+YEAR+>, <+AUTHOR+>. All rights reserved.
% 
% This file is license under the terms of the
%
\documentclass[a4paper,12pt, leqno, answers]{exam}
\usepackage[top=3cm, bottom=3cm, left=2cm, right=2cm]{geometry}
\usepackage[utf8]{inputenc}
\usepackage[brazil]{babel}
\usepackage{amsmath}
\usepackage{amsfonts}
\usepackage{hyperref}

% Customiza\c{c}\~{a}o da classe exam
\firstpageheader{MT402}{Solu\c{c}\~{a}o da Lista 6}{1º semestre de 2012}
\firstpageheadrule
\footer{Dispon\'{i}vel em \\\url{https://github.com/r-gaia-cs/solucoes_lista_matrizes}
}{}{Reportar erros para \\\href{mailto:r.gaia.cs@gmail.com}{r.gaia.cs@gmail.com}
}
\footrule 
\pagestyle{foot}
\renewcommand{\solutiontitle}{\noindent\textbf{Solu\c{c}\~{a}o:}\enspace}
\SolutionEmphasis{\itshape}
\unframedsolutions
\pointname{}

% Customiza\c{c}\~{a}o do pacote amsmath
\allowdisplaybreaks[4]

%Novos ambientes
\newenvironment{fwsolution}{\begin{EnvFullwidth}\begin{TheSolution}}{\end{TheSolution}\end{EnvFullwidth}}

% Novos comandos
%\newcommand{\mdot}{\text{\LARGE $\boldsymbol{\cdot}$}}
\newcommand{\mdot}{\bullet}

\begin{document}
\thispagestyle{headandfoot}
\begin{questions}
    \question A elimina\c{c}\~{a}o dos elementso $A_{ik}, i = k + 1, \ldots, n$ \'{e} realizada atrav\'{e}s de uma sequ\^{e}ncia de opera\c{c}\~{o}es elementares: multiplicar a linha $k$ por $m_{ik}$ e subtrair o resultado da linha $i$, $i = k + 1, \ldots, n$. Esta sequ\^{e}ncia de opera\c{c}\~{o}es comp\~{o}e a Transforma\c{c}\~{a}o de Gauss, $M_k$ da etapa $k$. Demonstre que:
    \begin{parts}
        \part Cada opera\c{c}\~{a}o elementar corresponde a pr\'{e}-multiplicar a matriz por $E_i = I - m_{ik} e_i e_k^t$;
        \begin{solution}
            Temos que
            \begin{align*}
                E_i A &= \left( I - m_{ik} e_i e_k^t \right) A \\
                &= A - m_{ik} e_i e_k^t A \\
                &= A - e_i \left( m_{ik} A_{k \mdot}^t \right).
            \end{align*}
        \end{solution}

        \part $M_k = E_n E_{n - 1} \ldots E_{k + 1} = I - \sum_{i = k + 1}^n m_{ik} e_i e_k^t = I - \gamma_k e_k^t$, onde $\gamma_k : n \times 1$, $\gamma_k(i) = 0, i = 1, \ldots, k$ e $\gamma_k(i) = m_{ik}, i = k + 1, \ldots, n$.
        \begin{solution}
            Temos que
            \begin{align*}
                M_k &= E_n E_{n - 1} \ldots E_{k + 1} \\
                &= \left( I - m_{n, k} e_n e_k^t \right) \left( I - m_{n - 1, k} e_{n-1} e_k^t \right) \ldots \left( I - m_{k + 1, k} e_{k + 1} e_k^t \right) \\
                % &= \left( I - m_{n - 1, k} e_{n - 1} e_k^t - m_{n, k} e_n e_k^t + m_{n, k} m_{n - 1, k} e_n e_k^t e_{n - 1} e_k^t \right) \ldots \left( I - m_{k + 1, k} e_{k + 1} e_k^t \right) \\
                &= \left( I - m_{n - 1, k} e_{n - 1} e_k^t - m_{n, k} e_n e_k^t \right) \ldots \left( I - m_{k + 1, k} e_{k + 1} e_k^t \right) && e_n e_k^t = 0 \\
                &= I - \sum_{i = k + 1}^n m_{ik} e_i e_k^t \\
                &= I - \left( \sum_{i = k + 1}^n m_{ik} e_i \right) e_k^t \\
                &= I - \gamma_k e_k^t,
            \end{align*}
            onde $\gamma_k : n \times 1$, $\gamma_k(i) = 0, i = 1, \ldots, k$ e $\gamma_k(i) = m_{ik}, i = k + 1, \ldots, n$.
        \end{solution}
    \end{parts}

    \question Demonstre que a matriz $L = \left( M_{n - 1} M_{n - 2} \ldots M_2 M_1 \right)^{-1} = I + \sum_{k = 1}^n \gamma_k e_k^t$, se os fatores $L$ e $U$ s\~{a}o calculados pelo processo de Elimina\c{c}\~{a}o de Gauss sem estrat\'{e}gia de pivoteamento parcial.
    \begin{solution}
        Temos que
        \begin{align*}
            L &= \left( M_{n - 1} M_{n - 2} \ldots M_2 M_1 \right)^{-1} \\
            &= M_1^{-1} M_2^{-1} \ldots M_{n - 2}^{-1} M_{n - 1}^{-1}
        \end{align*}
        Mas pelo exerc\'{i}cio 1(b) temos que $M_k = I - \gamma_k e_k^t$ de onde verificamos que
        \begin{align*}
            M_k M_k^{-1} &= \left( I - \gamma_k e_k^t \right) \left( I + \gamma_k e_k^t \right) \\
            &= I + \gamma+k e_k^t - \gamma_k e_k^t - \gamma_k e_k^t \gamma_k e_k^t \\
            &= I - \gamma_k e_k^t \gamma_k e_k^t \\
            &= I && e_k^t \gamma_k = 0.
        \end{align*}
        Logo,
        \begin{align*}
            L&= \left( I + \gamma_1 e_1^t \right) \left( I + \gamma_2 e_2^t \right) \ldots \left( I + \gamma_{n - 2} e_{n - 2}^t \right) \left( I + \gamma_{n - 1} e_{n - 1}^t \right) \\
            &= \left( I + \gamma_1 e_1^t + \gamma_2 e_2^t + \gamma_1 e_1^t \gamma_2 e_2^t \right) \ldots \left( I + \gamma_{n - 2} e_{n - 2}^t \right) \left( I + \gamma_{n - 1} e_{n - 1}^t \right) \\
            &= \left( I + \gamma_1 e_1^t + \gamma_2 e_2^t \right) \ldots \left( I + \gamma_{n - 2} e_{n - 2}^t \right) \left( I + \gamma_{n - 1} e_{n - 1}^t \right) \\
            &= I + \sum_{k = 1}^{n - 1} \gamma_k e_k^t.
        \end{align*}
    \end{solution}

    \question $P = I - u u^t$, $u = e_r - e_s$ representa a permuta\c{c}\~{a}o das linhas $r$ e $s$:
    \begin{parts}
        \part Demonstre que $P^{-1} = P$ e portanto, $P^2 = I$;
        \begin{solution}
            Se $P$ corresponde a permuta\c{c}\~{a}o das linhas $r$ e $s$ ent\~{a}o $P^{-1}$, a opera\c{c}\~{a}o inversa, corresponde a permuta\c{c}\~{a}o das linhas $s$ e $r$ que equivale a permuta\c{c}\~{a}o das linhas $r$ e $s$, logo $P^{-1} = P$.

            Portanto,
            \begin{align*}
                P P^{-1} &= P P \\
                &= \left( I - u u^t \right) \left( I - u u^t \right) \\
                &= I - u u^t - u u^t + u u^t u u^t \\
                &= I - 2 u u^t + \| u \|_2^2 u u^t \\
                &= I - 2 u u^t + 2 u u^t && u = e_r - e_s \\
                &= I.
            \end{align*}
        \end{solution}

        \part Demonstre que $P M_k P = I - \tilde{\gamma}_k e_k^t = M_k$ onde $\tilde{\gamma}_k$ \'{e} o vetor $\gamma_k$ com os multiplicadores $m_{rk}$ e $m_{sk}$, $r > k$ e $s > k$, permutados.
        \begin{solution}
            Pelo exerc\'{i}cio 1(b) temos que $M_k = I - \gamma_k e_k^t$ e portanto
            \begin{align*}
                P M_k P &= \left( I - u u^t \right) \left( I - \gamma_k e_k^t \right) \left( I - u u^t \right) \\
                &= \left( I - \gamma_k e_k^t - u u^t + u u^t \gamma_k e_k^t \right) \left( I - u u^t \right) \\
                &= \left( I - \gamma_k e_k^t - u u^t + \left( \gamma_k(r) - \gamma(s) \right) u e_k^t \right) \left( I - u u^t \right) \\
                \begin{split}
                    &= I - u u^t - \gamma_k e_k^t - \gamma_k e_k^t u u^t - u u^t + u u^t u u^t \\ &\quad {}+ \left( \gamma_k(r) - \gamma_k(s) \right) u e_k^t - \left( \gamma_k(r) - \gamma_k(s) \right) u e_k^t u u^t
                \end{split} \\
                \begin{split}
                    &= I - u u^t - \gamma_k e_k^t - u u^t + 2 u u^t + \left( \gamma_k(r) - \gamma_k(s) \right) u e_k^t
                \end{split} && e_k^t u = 0, r > k, s > k \\
                &= I - \gamma_k e_k^t + \left( \gamma_k(r) - \gamma_k(s) \right) u e_k^t \\
                &= I - \left( \gamma_k - \left( \gamma_k(r) - \gamma_k(s) \right) u \right) e_k^t \\
                &= I - \tilde{\gamma}_k e_k^t,
            \end{align*}
            onde $\tilde{\gamma}_k$ corresponde ao vetor $\gamma_k$ com os multiplicadores $m_{rk}$ e $m_{sk}$ permutados.
        \end{solution}
    \end{parts}

    \question[Ver Teorema 3.4.1, p\'{a}gina 113, do Golub\nocite{Golub:1996:matrix}] Se os fatores $L$ e $U$ s\~{a}o obtidos atrav\'{e}s da Elimina\c{c}\~{a}o de Gauss com pivoteamento parcial, represente $P$ e $L$ de acordo com as permuta\c{c}\~{o}es e transforma\c{c}\~{o}es de Gauss realizadas durante o processo.
    \begin{solution}
        Pelo exerc\'{i}cio 2 temos que, no caso em que n\~{a}o \'{e} utilizado permuta\c{c}\~{a}o,
        \begin{align*}
            L &= \left( M_{n - 1} M_{n - 2} \ldots M_2 M_1 \right)^{-1}.
        \end{align*}
        J\'{a} ao utilizarmos uma permuta\c{c}\~{a}o antes de cada opera\c{c}\~{a}o elementar temos
        \begin{align*}
            L &= \left( M_{n - 1} P_{n - 1} M_{n - 2} P_{n - 2} \ldots M_2 P_2 M_1 P_1 \right)^{-1},
        \end{align*}
        onde $P_k$ \'{e} a permuta\c{c}\~{a}o efetuada antes de realizar a opera\c{c}\~{a}o elementar $M_k$.
    \end{solution}

    \question Se $P A = L U$, como usar estes fatores para resolver o sistema linear $A^t x = b$?
    \begin{solution}
        Se $P A = L U$ ent\~{a}o $A = P^t L U$. Logo,
        \begin{align*}
            A^t x &= \left( P^t L U \right) x \\
            &= U^t L^t P x.
        \end{align*}
        E para resolver o sistema utilizamos $x = P^t L^{-t} U^{-t} b$.
    \end{solution}

    \question Descreva como o processo de elimina\c{c}\~{a}o de Gauss com pivoteamente parcial pode ser usando para obter o determinante de uma matriz $A : n \times n$.
    \begin{solution}
        Seja $P A = L U$, ent\~{a}o
        \begin{align*}
            \det(P A) &= \det(L U) \\
            &= \det(L) \det (U) \\
            &= \prod_{i = 1}^n l_{ii} \prod_{i = 1}^n u_{ii} && \text{pois L e U s\~{a}o diagonais} \\
            &= \prod_{i = 1}^n u_{ii} && \text{pois L possui diagonal unit\'{a}ria}.
        \end{align*}
        Mas
        \begin{align*}
            \det(P A) &= \det(P) \det(A) \\
            &= (-1)^{c} \det(A),
        \end{align*}
        onde $c$ \'{e} o n\'{u}mero de permuta\c{c}\~{o}es realizadas. Logo, $\det(A) = (-1)^c \prod_{i = 1}^n u_{ii}$.
    \end{solution}

    \question(Ver equa\c{c}\~{a}o 3.10.12, p\'{a}gina 149, do Meyer\nocite{Meyer:2000:matrix}] Sobre o teorema da exist\^{e}ncia e unicidade da fatora\c{c}\~{a}o LU: se uma submatriz principal dominante de ordem $k$ for singular, podemos afirmar que a matriz $A$ n\~{a}o tem fatora\c{c}\~{a}o $LU$? Fundamente sua resposta teoricamente e com exemplos.
    \begin{solution}
        O Teorema 3.2.1, p\'{a}gina 97, do Golub\nocite{Golub:1996:matrix} trata da exist\^{e}ncia e unicidade da fatora\c{c}\~{a}o LU e \'{e} transcrito abaixo:
        \begin{quote}
            $A \in \mathbb{R}{^n \times n}$ possue fatora\c{c}\~{a}o LU se $\det(A(1:k, 1:k)) \neq 0$ para $k = 1:n - 1$. Se a fatora\c{c}\~{a}o LU existe e $A$ \'{e} n\~{a}o singular, ent\~{a}o a fatora\c{c}\~{a}o LU \'{e} \'{u}nica e $\det(A) = \prod_{i = 1}^n u_{[ii}$.
        \end{quote}

        Como podemos notar, o teorema apenas garante que se todas as submatrizes principais s\~{a}o n\~{a}o singulares ent\~{a}o a fatora\c{c}\~{a}o LU existe e nada \'{e} afirmada para o caso de alguma das submatrizes principais ser singular.

        Pelo contra-exemplo abaixo,
        \begin{align*}
            A = \begin{bmatrix}
                1 & 1 \\
                1 & 0
            \end{bmatrix} = \begin{bmatrix}
                1 & 0 \\
                1 & 1
            \end{bmatrix} \begin{bmatrix}
                1 & 1 \\
                0 & -1
            \end{bmatrix} = L U,
        \end{align*}
        mostramos que mesmo existindo uma submatriz singular a fatora\c{c}\~{a}o LU ainda existe.
    \end{solution}

    \question Descreva como obter a inversa de uma matriz $A$ atrav\'{e}s da resolu\c{c}\~{a}o de $n$ sistemas lineares. A fatora\c{c}\~{a}o $LU$ com pivoteamento parcial \'{e} indicada para esta resolu\c{c}\~{a}o? Qual o número total de opera\c{c}\~{o}es necess\'{a}rias para obter $A^{-1}$?
    \begin{solution}
        Seja $B = A^{-1}$. Podemos obter a matriz $B$ resolvendo os seguintes sistemas lineares:
        \begin{align*}
            A B_{\mdot i} = e_i, \forall i \in \left\{ 1, 2, \ldots, n \right\}.
        \end{align*}
        A fatora\c{c}\~{a}o LU com pivoteamento parcial \'{e} indicada para esta resolu\c{c}\~{a}o pois temos que resolver $n$ sistemas lineares que diferem apenas no lado direito e ao utilizarmos pivoteamento parcial temos garantia de encontrar a fatora\c{c}\~{a}o LU da matriz $A$.

        Para obter a fatora\c{c}\~{a}o LU necessitamos de $2 n^3 / 3$ opera\c{c}\~{o}es e para resolver um sistema triangular de $n^2$ opera\c{c}\~{o}es. Como para obter a matriz $B$ efetuamos uma fatora\c{c}\~{a}o LU e $2 n$ resolu\c{c}\~{o}es de sistemas triangulares temos que o n\'{u}mero total de opera\c{c}\~{o}es \'{e} $2 n^3 / 3 + 2 n^3$.
    \end{solution}

    \question Supor que s\~{a}o dados $A : n \times n$, $d: n \times 1$ e $c : n \times 1$, e o objetivo \'{e} calcular: $z = c^t A^{-1} d$. Na express\~{a}o de $z$ a dificuldade est\'{a} no c\'{a}lculo de $s = A^{-1} d$. Analise os dois processos para obter $s$:
    \begin{enumerate}
        \item invertendo a matriz $A$ e calculando $s = A^{-1} d$ e
        \item resolvendo o sistema linear $A s = d$.
    \end{enumerate}
    Qual das duas formas deve ser escolhida de modo que $z$ seja obtido de modo econômico? Conclus\~{a}o: \'{e} melhor resolver um sistema linear do que inverter uma matriz? Justifique!
    \begin{solution}
        Deve escolher resolver o sistema linear $A s = d$ pois este processo requer um n\'{u}mero de opera\c{c}\~{o}es pr\'{o}ximo de $2 n^3 / 3$ (realizando uma fatora\c{c}\~{a}o LU e a resolu\c{c}\~{a}o de um sistema linear) enquanto que apenas inverter a matriz $A$ requer um n\'{u}mero de opera\c{c}\~{o}es pr\'{o}ximo de $7 n^3 / 3$ opera\c{c}\~{o}es (realizando uma fatora\c{c}\~{a}o LU e a resolu\c{c}\~{a}o de $n$ sistemas lineares, ver exerc\'{i}cio 8).
    \end{solution}

    \question Supor que a matriz $A : n \times n$ \'{e} particionada na forma
    \[
    A = \begin{bmatrix}
        A_{11} & A_{12} \\
        A_{12} & A_{22}
    \end{bmatrix}
    \]
    onde $A_{11} : r \times s$ \'{e} n\~{a}o singular. Denote por $S$, o complemento de Schr de $A_{11}$ em $A$. Demonstre que $A$ \'{e} n\~{a}o singular, se e somente se $S = A_{22} - A_{21} A_{11}^{-1} A_{12}$ \'{e} n\~{a}o singular. Sugest\~{a}o: escreva a fatora\c{c}\~{a}o $LU$ de $A$ em fun\c{c}\~{a}o das submatrizes $A_{ij}$.
    \begin{solution}
        Temos que a fatora\c{c}\~{a}o LU da matriz $A$ em termos de sua submatrizes \'{e} dado por
        \begin{align*}
            A = \begin{bmatrix}
                A_{11} & A_{12} \\
                A_{21} & A_{22}
            \end{bmatrix} = \begin{bmatrix}
                I & 0 \\
                -A_{21} A_{11}^{-1} & I
            \end{bmatrix} \begin{bmatrix}
                A_{11} & A_{12} \\
                0 & A_{22} - A_{21} A_{11}^{-1} A_{12}
            \end{bmatrix},
        \end{align*}
        onde $A_{22} - A_{21} A_{11}^{-1} A_{12} = S$ \'{e} o complemento de Schur de $A_{11}$.

        Para que $A$ seja n\~{a}o singular devemos ter que $\det(A) \neq 0$. Logo,
        \begin{align*}
            \det(A) &= \det(L U) \\
            &= \det(L) \det(U) \\
            &= 1 \det(U) && \text{L \'{e} triangular unit\'{a}ria} \\
            &= \det(A_{11}) \det(S) && \text{U \'{e} triangular}.
        \end{align*}
        Portanto, $A$ \'{e} n\~{a}o singular se e somente se $S$ \'{e} n\~{a}o singular.
    \end{solution}

    \question[Teorema 3.4.3, p\'{a}gina 120, do Golub\nocite{Golub:1996:matrix}] Considere $A : n \times n$, estritamente diagonal dominante por colunas, isto \'{e}: $| a_{ij} | > \sum_{j \neq k} | a_{jk} |$. Demonstre que a fatora\c{c}\~{a}o $LU$ com pivoteamente parcial aplicada sobre a matriz $A$, n\~{a}o ir\'{a} realizar nenhuma permuta\c{c}\~{a}o de linhas, ou seja, ao final do processo a matriz $P$ ser\'{a} a identidade.
    \begin{solution}
        Particionando a matriz $A$ como
        \begin{align*}
            A = \begin{bmatrix}
                \alpha & w^t \\
                v & C
            \end{bmatrix},
        \end{align*}
        onde $\alpha \in \mathbb{R}$. Ap\'{o}s uma itera\c{c}\~{a}o da fatora\c{c}\~{a}o LU temos
        \begin{align*}
            \begin{bmatrix}
                \alpha & w^t \\
                v & C
            \end{bmatrix} = \begin{bmatrix}
                1 & 0 \\
                v / \alpha & 1
            \end{bmatrix} \begin{bmatrix}
                1 & 0 \\
                0 & C - v w^t / \alpha
            \end{bmatrix} \begin{bmatrix}
                \alpha & w^t \\
                0 & I
            \end{bmatrix}.
        \end{align*}
        Segue por indi\c{c}\~{a}o em $n$ que se $B = C - v w^t / \alpha$ \'{e} estritamente diagonal dominante ent\~{a}o $A$ possue fatora\c{c}\~{a}o LU, pois se $B = L_1 U_1$ ent\~{a}o
        \begin{align*}
            A = \begin{bmatrix}
                1 & 0 \\
                v / \alpha & L_1
            \end{bmatrix} \begin{bmatrix}
                \alpha & w^t \\
                0 & U_1
            \end{bmatrix} = L U.
        \end{align*}
        Para provar que $B$ \'{e} estritamente diagonal dominante, segue da defini\c{c}\~{a}o que
        \begin{align*}
            \sum_{i = 1, i \neq j}^{n - 1} | b_{ij} | &= \sum_{i = 1, i \neq j}^{n - 1} | c_{ij} - v_i w_j / \alpha | \\
            &\leq \sum_{i = 1, i \neq j}^{n - 1} | c_{ij} | + \left( | w_j | / | \alpha | \right) \sum_{i = 1, i \neq j}^{n - 1} | v_i | && \text{desigualdade triangular} \\
            &\leq \left( | c_{jj} | - | w_j | \right) + \left( | w_j | / | \alpha | \right) \left( | \alpha | - | v_j | \right) && \text{pois $A$ \'{e} diagonalmente dominante} \\
            &\leq \left\| c_{jj} - w_j v_j / \alpha \right\| \\
            &= \| b_{jj} \|.
        \end{align*}
    \end{solution}

    \question Considere o sistema linear $A x = b$, $A : n \times n$ n\~{a}o singular. O objetivo \'{e} analisar as solu\c{c}\~{o}es dos sistemas: $A x = b$ e $A (x + \delta x) = b + \delta b$, (introduzimos uma pertuba\c{c}\~{a}o no vetor $b$). \'{E} desej\'{a}vel que se esta pertuba\c{c}\~{a}o for pequena ent\~{a}o a pertuta\c{c}\~{a}o na solu\c{c}\~{a}o tamb\'{e}m seja pr\'{o}xima de zero. Analisando em termos relativos aos vetores originais estamos perguntando se vale a rela\c{c}\~{a}o: $\| \delta b\| / \| b \| \approx 0 \rightarrow \| \delta x \| / \| x \| \approx 0$.
    \begin{parts}
        \part Demonstre que $\| \delta x \| \| x \| \leq \text{cond}(A) \| \delta b \| / \| b \|$ e analise se a rela\c{c}\~{a}o acima \'{e} verdadeira ou falsa.
        \begin{solution}
            
        \end{solution}
    \end{parts}
    
    \question[Trefethen\nocite{Trefethen:1997:numerical}, Lecture 22] Sobre o Fator de Crescimento:
    \begin{quote}
        \textbf{Teorema:} Supor que a fatora\c{c}\~{a}o $LU$ de $A$ foi obtida sem pivoteamento. Ent\~{a}o para todo $\epsilon_\text{maq}$, precis\~{a}o da m\'{a}quina, suficientemente pr\'{o}ximo de zero, a fatora\c{c}\~{a}o \'{e} realizada com sucesso em aritm\'{e}tica de ponto flutuante (nenhum pivô nulo \'{e} gerado) e as matrizes computadas, $\tilde{L}$ e $\tilde{U}$ satisfazem a equa\c{c}\~{a}o:
        \[
        \tilde{L} \tilde{U} = A + \delta A, \frac{\| \delta A \|}{\| L \| \| U \|} = \mathcal{O}(\epsilon_\text{maq}).
        \]
        (Informa\c{c}\~{o}es: $\epsilon_\text{maq} \approx 10^{-16}$ em precis\~{a}o dupla e $\epsilon_\text{maq} \approx 10^{-8}$ em precis\~{a}o simples.)
    \end{quote}
    Considerando agora a fatora\c{c}\~{a}o $LU$ de $A$ com pivoteamento parcial. Neste caso, $\| L \| = \mathcal{O}(1)$ (por qu\^{e}?). O fator de crescimento para $A$ \'{e} definido por $\rho = \max u_{ij} / \max a_{ij}$ e usando este fator e a rela\c{c}\~{a}o do teorema podemos extrair a rela\c{c}\~{a}o: $\| U \| = \mathcal{O}(\rho \| A \|)$ (deduza essa rela\c{c}\~{a}o). E, da\'{i} temos finalmente a rela\c{c}\~{a}o:
    \[
    \tilde{L} \tilde{U} = P A + \delta A, \frac{\| \delta A \|}{\| A \|} = \mathcal{O}(\rho \epsilon_\text{maq}).
    \]
    Portanto, podemos afirmar que a fatora\c{c}\~{a}o $LU$ \'{e} backward stable\footnote{Estabilidade analisada quando se compara a matriz original com a obtida ao se multiplicar $L$ por $U$.} se $\rho = \mathcal{O}(1)$.
    \begin{parts}
        \part Demonstre que $\rho \leq 2^{n-1}$ quando os fatores $L$ e $U$ s\~{a}o obtidos com estrat\'{e}gia de pivoteamento parcial. Conclua que este limite superior pode resultar em um fator de crescimento excessiamente grande para matrizes de grande porte.
        \begin{solution}
            
        \end{solution}

        \part Efetue a fatora\c{c}\~{a}o $LU$ de $A : 5 \times 5$ com pivoteamento parcial e calcule o fator de crescimento, onde
        \[
        A = \begin{bmatrix}
            1 & 0 & 0 & 0 & 1 \\
            -1 & 1 & 0 & 0 & 1 \\
            -1 & -1 & 1 & 0 & 1 \\
            -1 & - 1 & -1 & 1 & 1 \\
            -1 & -1 & -1 & -1 & 1
        \end{bmatrix}
        \]
        e verifique que este limitante superior para $\rho$ pode ocorrer\footnote{O exemplo foi constru\'{i}do para provar que o pior caso pode acontecer. Em aplica\c{c}\~{o}es reais, tal fator de crescimento nunca ocorreu.}.
        \begin{solution}
            
        \end{solution}
    \end{parts}
\end{questions}
\bibliographystyle{plain}
\bibliography{bibliography}
\end{document}
