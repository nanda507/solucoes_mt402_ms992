% Filename: lista07.tex
% This code is part of 'Listas de Matrizes'
% 
% Description: Lista 7.
% 
% Created: 27.04.12 10:44:51 AM
% Last Change: 29.04.12 08:08:58 PM
% 
% Author: <+AUTHOR+>, <+EMAIL+>
% Organization: <+ORGANIZATION+>
% 
% Copyright (c) <+YEAR+>, <+AUTHOR+>. All rights reserved.
% 
% This file is license under the terms of
%
\documentclass[a4paper,12pt, leqno, answers]{exam}
\usepackage[top=3cm, bottom=3cm, left=2cm, right=2cm]{geometry}
\usepackage[utf8]{inputenc}
\usepackage[brazil]{babel}
\usepackage{amsmath}
\usepackage{amsfonts}
\usepackage{hyperref}
\usepackage{algorithmicx}
\usepackage{algpseudocode}

% Customiza\c{c}\~{a}o da classe exam
\firstpageheader{MT402}{Solu\c{c}\~{a}o da Lista 7}{1º semestre de 2012}
\firstpageheadrule
\footer{Dispon\'{i}vel em \\\url{https://github.com/r-gaia-cs/solucoes_lista_matrizes}
}{}{Reportar erros para \\\href{mailto:r.gaia.cs@gmail.com}{r.gaia.cs@gmail.com}
}
\footrule 
\pagestyle{foot}
\renewcommand{\solutiontitle}{\noindent\textbf{Solu\c{c}\~{a}o:}\enspace}
\SolutionEmphasis{\itshape}
\unframedsolutions
\pointname{}

% Customiza\c{c}\~{a}o do pacote amsmath
\allowdisplaybreaks[4]

%Novos ambientes
\newenvironment{fwsolution}{\begin{EnvFullwidth}\begin{TheSolution}}{\end{TheSolution}\end{EnvFullwidth}}

% Novos comandos
%\newcommand{\mdot}{\text{\LARGE $\boldsymbol{\cdot}$}}
\newcommand{\mdot}{\bullet}

\begin{document}
\thispagestyle{headandfoot}
\begin{questions}
    \question[Ver exerc\'{i}cio 3.10.9, p\'{a}gina 157, Meyer\nocite{Meyer:2000:matrix}] Demonstre que se a matriz $A : n \times n$ tem fatora\c{c}\~{a}o LU ent\~{a}o \'{e} poss\'{i}vel obter uma fatora\c{c}\~{a}o $A = L D \tilde{U}$ com $\tilde{u}_{ii} = 1$. Se, $A$ for tamb\'{e}m sim\'{e}trica ent\~{a}o esta fatora\c{c}\~{a}o \'{e} da forma $A = L D L^t$.
    \begin{solution}
        Se a matriz $A$ tem fatora\c{c}\~{a}o LU ent\~{a}o $A = LU$ e \'{e} poss\'{i}vel obter uma fatora\c{c}\~{a}o $A = L D \tilde{U}$ com $\tilde{u}_{ii} = 1$ pois $d_{ii} = u_{ii}$, $d_{ij} = 0$ para $i \neq j$ e $\tilde{u}_{ij} = u_{ij} / u_{ii}$ para $j > i$.
    \end{solution}

    \question Considere $C : n \times n$ uma matriz com posto completo ($\text{posto}(C) = n$). Demonstre que as matrizes $C^t C$ e $C C^t$ s\~{a}o sim\'{e}tricas definidas positivas.
    \begin{solution}
        As matrizes $C^t C$ e $C^t C$ s\~{a}o sim\'{e}tricas definidas positivas se e somente se $x^t C^t C x \geq 0$ e $x^t C C^t x \geq 0$ para todo $x$ sendo que a igualdade ocorre apenas para $x = 0$. Ent\~{a}o $C x \geq 0$ e $C^t x \geq 0$ sendo que a igualdade ocorre apenas para $x = 0$. Como $C$ tem posto completo, verificamos que as desigualdades anteriores s\~{a}o verificadas.
    \end{solution}

    \question Demonstre que a soma de duas matrizes definidas positiva \'{e} uma matriz definida positiva.
    \begin{solution}
        Seja $A$ e $B$ duas matrizes definidas positivas, i.e., $x^t A x \geq 0$ e $x^t B x \geq 0$ sendo que a igualdade \'{e} verificada apenas para $x = 0$. Ent\~{a}o,
        \begin{align*}
            x^t \left( A + B \right) x &= x^t \left( A x + B x \right) \\
            &= x^t A x + x^t B x.
        \end{align*}
        Logo, a soma de duas matrizes definidas positivas \'{e} definida positiva.
    \end{solution}

    \question Demonstre que se $A : n \times n$ \'{e} definida positiva, ent\~{a}o $\max_{ij} \| a_{ij} | = \max_i a_{ij} > 0$. (Sugest\~{a}o: escolha $x = e_r - \text{sign}(a_{rs}) e_s$ onde $| a_{rs} | = \max_{ij} | a_{ij} |$ e analise os casos $r = s$ e $r \neq s$.)
    \begin{solution}
        Seja $x = e_r - \text{sign}(a_{rs}) e_s$, ent\~{a}o
        \begin{align*}
            x^t A x &= \left( e_r - \text{sign}(a_{rs}) e_s \right)^t A \left( e_r - \text{sign}(a_{rs}) e_s \right) \\
            &= \left( e_r - \text{sign}(a_{rs}) e_s \right)^t \left( A e_r - \text{sign}(a_{rs}) A e_s \right) \\
            &= e_r^t A e_r - \text{sign}(a_{rs}) e_r^t A e_s - \text{sign}(a_{rs} e_s^t A e_r + \text{sign}(a_{rs}) \text{sign}(a_{rs}) e_s^t A e_s \\
            &= a_{rr} - \text{sign}(a_{rs}) \left( a_{rs} + a_{sr} \right) + a_{ss}.
        \end{align*}
        Como $A$ \'{e} definida positiva $a_{rr} - \text{sign}(a_{rs}) \left( a_{rs} + a_{sr} \right) + a_{ss} \geq 0$. E como $A$ \'{e} sim\'{e}trica, $a_{rs} = a_{sr}$, temos que
        \begin{align*}
            \text{sign}(a_{rs}) a_{r1} \leq \left( a_{rr} + a_{ss} \right) / 2 \leq \max_i a_{ii}
        \end{align*}
        e portanto $\max_{i,j} | a_{ij} | = \max_i a_{ii}$.
    \end{solution}

    \question Demonstre que se $A : n \times n$ \'{e} definida positiva, ent\~{a}o suas submatrizes principais dominantes de ordem $k$, $A_k$, $k = 1, 2, \ldots, n$ s\~{a}o matrizes definidas positivas.
    \begin{solution}
        Seja
        \begin{align*}
            A &= \begin{bmatrix}
                A_{11} & A_{12} \\
                A_{21} & A_{22}
            \end{bmatrix}
        \end{align*}
        uma matriz definida positiva, i.e., $x^t A x \geq 0$ para qualquer $x$. Tomemos $x = (x_1^t, 0)^t$, ent\~{a}o
        \begin{align*}
            x^t A x &= \begin{bmatrix}
                x_1^t & 0
            \end{bmatrix} \begin{bmatrix}
                A_{11} & A_{12} \\
                A_{21} & A_{22}
            \end{bmatrix} \begin{bmatrix}
                x_1 \\
                0
            \end{bmatrix} \\
            &= \begin{bmatrix}
                x_1^t & 0
            \end{bmatrix} \begin{bmatrix}
                A_{11} x_1 \\
                A_{21} x_1
            \end{bmatrix} \\
            &= x_1^t A_{11} x_1.
        \end{align*}
        Como $x^t A x \geq 0$ para qualquer $x$ temos que $x_1^t A_{11} x_1 \geq 0$ para qualquer $x_1$ e portanto as submatrizes principais dominantes de ordem $k$ s\~{a}o matrizes definidas positivas.
    \end{solution}

    \question Julgue verdadeiro ou Falso: Se $A$ \'{e} definida positiva ent\~{a}o $\det(A) > 0$.
    \begin{solution}
        A matrix $A$, como toda matriz quadrada, possue uma base de autovetores que fomam a matriz $P$. Calculando $P A P^{-1}$ obtemos a matriz na sua forma normal de Jordan, $J$, que \'{e} triangular superior. Sabemos que
        \begin{align*}
            \det(J) &= \det(P A P^{-1}) \\
            &= \det(P) \det(A) \det(P^{-1}) \\
            &= \det(A) && \det(P^{-1}) = 1 / \det(P).
        \end{align*}
        e que $\det(J)$ \'{e} o produto dos elementos da diagonal principal. Os elementos da diagonal principal da matriz $J$ s\~{a}o os autovalores da matrix, ent\~{a}o o determinante de $A$ \'{e} o produto dos seus autovalores.

        Suponha que $c$ \'{e} um autovalor positivo, ent\~{a}o existe um autovetor $v$ tal que
        \begin{align*}
            A v &= c v \\
            v^t A v &= c v^t v.
        \end{align*}
        Note que $v^t v$ \'{e} a norma-2 de $v$ que \'{e} n\~{a}o negativa e zero se e somente se $v = 0$. Mas como $v$ \'{e} um autovetor sabemos que $v \neq 0$ e portanto $v^t v > 0$.

        Como $v^t A v > 0$ pois a matriz \'{e} definida positiva e $v^t v > 0$ concluimos que $c > 0$.

        Portanto, todo autovalor real \'{e} positivo e os autovalores complexos aparecem juntamente com seu conjugado. Logo, $\det(A) > 0$.
    \end{solution}

    \question[Exemplo 3.10.7, p\'{a}gina 154, do Meyer\nocite{Meyer:2000:matrix}] Demonstre que $A : n \times n$, sim\'{e}trica, \'{e} definida positiva, se e somente se $A$ tem fatora\c{c}\~{a}o de Cholesky.
    \begin{solution}
        Se $A$ \'{e} definida positiva ent\~{a}o existe uma fatora\c{c}\~{a}o LDU tal que $A = L D L^t$, onde $D = \text{diag}(p_i, p_2, \ldots, p_n)$ \'{e} uma matriz diagonal contendo os pivôs $p_i > 0$. Fazendo $R = D^{1/2} L^t$ onde $D^{1/2} = \text{diag}(\sqrt{p_1}, \sqrt{p_2}, \ldots, \sqrt{p_n})$ encontramos a fatora\c{c}\~{a}o desejada pois $A = L D^{1/2} D^{1/2} L^t = R^t R$, e $R$ \'{e} uma matriz triangular superior com diagonal positiva.
        
        E se $A = R R^t$, onde $R$ \'{e} uma matriz triangular inferior com diagonal positiva, ent\~{a}o fatorando apenas a diagonal de $R$ obtemos $R = L D$ onde $L$ \'{e} triangular inferior com diagonal unit\'{a}ria e $D$ \'{e} uma matriz diagonal cujas entras s\~{a}o $r_{ii}$. Consequentemente, $A = L D^2 L^t$ \'{e} a fatora\c{c}\~{a}o LDU de $A$, e portanto os pivôs s\~{a}o positivos pois eles s\~{a}o os elementos da diagonal de $D^2$.
    \end{solution}

    \question Julgue verdadeiro ou Falso: Se $A : n \times n$ \'{e} definida positiva ent\~{a}o o sistema linear $A x = b$ admite solu\c{c}\~{a}o e esta solu\c{c}\~{a}o \'{e} \'{u}nica.
    \begin{solution}
        Suponha que existem duas solu\c{c}\~{o}es distintas para o sistema $A x = b$ dadas por $x_1$ e $x_2$ ($x_1 \neq x_2$), com $A$ definida positiva. Ent\~{a}o
        \begin{align*}
            A x_1 &= b, \\
            A x_2 &= b.
        \end{align*}
        Logo,
        \begin{align*}
            A x_1 - A x_2 &= b - b = 0 \\
            A (x_1 - x_2) &= 0.
        \end{align*}
        Como $A$ \'{e} definida positiva, $(x_1 - x_2)^t A (x_1 - x_2) \geq 0$ e $(x_1 - x_2)^t A (x_1 - x_2) = 0$ se e somente se $(x_1 - x_2) = 0$. Mas sabemos que $(x_1 - x_2)^t A (x_1 - x_2) = (x_1 - x_2)^t 0 = 0$, portanto $(x_1 - x_2) = 0$ que corresponde a $x_1 = x_2$ (contradi\c{c}\~{a}o).
    \end{solution}

    \question Demonstre que se $A$ \'{e} sim\'{e}trica definida positiva, ent\~{a}o seus autovalores s\~{a}o reais e estritamente positivos.
    \begin{solution}
        Suponha que $c$ \'{e} um autovalor de $A$, ent\~{a}o existe um autovetor $v$ tal que
        \begin{align*}
            A v &= c v \\
            v^t A v &= c v^t v.
        \end{align*}
        Note que $v^t v$ \'{e} a norma-2 de $v$ que \'{e} n\~{a}o negativa e zero se e somente se $v = 0$. Mas como $v$ \'{e} um autovetor sabemos que $v \neq 0$ e portanto $v^t v > 0$.

        Como $v^t A v > 0$ pois a matriz \'{e} definida positiva e $v^t v > 0$ concluimos que $c > 0$.
    \end{solution}

    \question Considere a matriz
    \begin{align*}
        A = \begin{bmatrix}
            9 & -6 & 12 & 9 \\
            -6 & 13 & -11 & -21 \\
            12 & -11 & 21 & 21 \\
            9 & -21 & 21 & 63
        \end{bmatrix}.
    \end{align*}
    Obtenha o fator de Cholesky de $A$ aplicando o algoritmo padr\~{a}o e a vers\~{a}o que utiliza o produto externo. Obtidos os fatores, resolva o sistema $A x = b$, com $b = \left( 48, -44, 76, 51 \right)^t$.
    \begin{solution}
        O Algoritmo 4.2.1, p\'{a}gina 144, do Golub\nocite{Golub:1996:matrix} corresponde a fatora\c{c}\~{a}o de Cholesky padr\~{a}o e \'{e} transcrito abaixo:
        \begin{algorithmic}
            \For{$j = 1, \ldots, n$}
                \If{$j > 1$}
                    \State $A(j:n, j) = A(j:n, j) - A(j:n, 1:j - 1) A(j, 1:j - 1)^t$
                \EndIf
                \State $A(j:n, j) = A(j:n, j) / \sqrt{A(j,j)}$
            \EndFor
        \end{algorithmic}

        Ent\~{a}o,
        \begin{enumerate}
            \item $j = 1$,
                \begin{align*}
                    A^{(1)} &\leftarrow \begin{bmatrix}
                        3 & -6 & 12 & 9 \\
                        -2 & 13 & -11 & -21 \\
                        4 & -11 & 21 & 21 \\
                        3 & -21 & 21 & 63
                    \end{bmatrix};
                \end{align*}
            \item $j = 2$,
                \begin{align*}
                    A^{(2)} &\leftarrow \begin{bmatrix}
                        3 & -6 & 12 & 9 \\
                        -2 & 13 - (-2)(-2) & -11 & -21 \\
                        4 & -11 - 4(-2) & 21 & 21 \\
                        3 & -21 - 3(-2) & 21 & 63
                    \end{bmatrix} = \begin{bmatrix}
                        3 & -6 & 12 & 9 \\
                        -2 & 9 & -11 & -21 \\
                        4 & -3 & 21 & 21 \\
                        3 & -15 & 21 & 63
                    \end{bmatrix} \\
                    A^{(2)} &\leftarrow \begin{bmatrix}
                        3 & -6 & 12 & 9 \\
                        -2 & 3 & -11 & -21 \\
                        4 & -1 & 21 & 21 \\
                        3 & -5 & 21 & 63
                    \end{bmatrix};
                \end{align*}
           \item  $j = 3$,
               \begin{align*}
                   A^{(3)} &\leftarrow \begin{bmatrix}
                        3 & -6 & 12 & 9 \\
                        -2 & 3 & -11 & -21 \\
                        4 & -1 & 21 - (4(4) + (-1)(-1))  & 21 \\
                        3 & -5 & 21 - (3(4) + (-5)(-1)) & 63
                    \end{bmatrix} = \begin{bmatrix}
                        3 & -6 & 12 & 9 \\
                        -2 & 3 & -11 & -21 \\
                        4 & -1 & 4  & 21 \\
                        3 & -5 & 4 & 63
                    \end{bmatrix} \\
                    A^{(3)} &\leftarrow \begin{bmatrix}
                        3 & -6 & 12 & 9 \\
                        -2 & 3 & -11 & -21 \\
                        4 & -1 & 2  & 21 \\
                        3 & -5 & 2 & 63
                    \end{bmatrix};
               \end{align*}
           \item $j = 4$,
               \begin{align*}
                   A^{(4)} &\leftarrow \begin{bmatrix}
                        3 & -6 & 12 & 9 \\
                        -2 & 3 & -11 & -21 \\
                        4 & -1 & 2  & 21 \\
                        3 & -5 & 2 & 63 - (3^2 + (-5)^2 + 2^2)
                    \end{bmatrix} = \begin{bmatrix}
                        3 & -6 & 12 & 9 \\
                        -2 & 3 & -11 & -21 \\
                        4 & -1 & 2  & 21 \\
                        3 & -5 & 2 & 25
                    \end{bmatrix} \\
                    A^{(4)} &\leftarrow \begin{bmatrix}
                        3 & -6 & 12 & 9 \\
                        -2 & 3 & -11 & -21 \\
                        4 & -1 & 2  & 21 \\
                        3 & -5 & 2 & 5
                    \end{bmatrix}
               \end{align*}
        \end{enumerate}
        Logo,
        \begin{align*}
            R &= \begin{bmatrix}
                3 & 0 & 0 & 0 \\
                -2 & 3 & 0 & 0 \\
                4 & -1 & 2  & 0\\
                3 & -5 & 2 & 5
            \end{bmatrix}.
        \end{align*}
        Resolvendo o sistema $R y = b$ temos que $y = (16.00, 25.33, 18.67, 18.47)^t$ e resolvendo o sistema $R^t x = y$ temos que $x = (5.11, 16.48, 5.64, 3.69)^t$.
        
        O Algoritmo 4.2.2, p\'{a}gina 145, do Golub\nocite{Golub:1996:matrix} corresponde a fatora\c{c}\~{a}o de Cholesky utilizando o produto externo e \'{e} transcrito abaixo:
        \begin{algorithmic}
            \For{$k = 1, \ldots, n$}
                \State $A(k,k) = \sqrt{A(k,k)}$
                \State $A(k + 1:n, k) = A(k + 1:n, k) / A(k,k)$
                \For{$j = k + 1:n$}
                    \State $A(j:n, j) = A(j:n, j) - A(j:n, k) A(j, k)$
                \EndFor
            \EndFor
        \end{algorithmic}

        Ent\~{a}o,
        \begin{enumerate}
            \item $k = 1$,
                \begin{align*}
                    A^{(1)} &\leftarrow \begin{bmatrix}
                        3 & -6 & 12 & 9 \\
                        -2 & 13 - (-2)(-2) & -11 & -21 \\
                        4 & -11 - (-2)4 & 21 - 4(4) & 21 \\
                        3 & -21 -(-2)3 & 21 - 4(3) & 63 - 3(3)
                    \end{bmatrix} = \begin{bmatrix}
                        3 & -6 & 12 & 9 \\
                        -2 & 9 & -11 & -21 \\
                        4 & -3 & 5 & 21 \\
                        3 & -15 & 9 & 54
                    \end{bmatrix}
                \end{align*}
            \item $k = 2$,
                \begin{align*}
                    A^{(2)} &\leftarrow \begin{bmatrix}
                        3 & -6 & 12 & 9 \\
                        -2 & 3 & -11 & -21 \\
                        4 & -1 & 5 - (-1)(-1) & 21 \\
                        3 & -5 & 9 - (-5)(-1) & 54 - (-5)(-5)
                    \end{bmatrix} = \begin{bmatrix}
                        3 & -6 & 12 & 9 \\
                        -2 & 3 & -11 & -21 \\
                        4 & -1 & 4 & 21 \\
                        3 & -5 & 4 & 29
                    \end{bmatrix}
                \end{align*}
           \item  $k = 3$,
               \begin{align*}
                   A^{(3)} &\leftarrow \begin{bmatrix}
                        3 & -6 & 12 & 9 \\
                        -2 & 3 & -11 & -21 \\
                        4 & -1 & 2 & 21 \\
                        3 & -5 & 2 & 29 - 2(2)
                    \end{bmatrix} = \begin{bmatrix}
                        3 & -6 & 12 & 9 \\
                        -2 & 3 & -11 & -21 \\
                        4 & -1 & 2 & 21 \\
                        3 & -5 & 2 & 25
                    \end{bmatrix}
               \end{align*}
           \item $k = 4$,
               \begin{align*}
                   A^{(4)} &\leftarrow \begin{bmatrix}
                        3 & -6 & 12 & 9 \\
                        -2 & 3 & -11 & -21 \\
                        4 & -1 & 2 & 21 \\
                        3 & -5 & 2 & 5
                    \end{bmatrix}
               \end{align*}
        \end{enumerate}
        Logo,
        \begin{align*}
            R &= \begin{bmatrix}
                3 & 0 & 0 & 0 \\
                -2 & 3 & 0 & 0 \\
                4 & -1 & 2  & 0\\
                3 & -5 & 2 & 5
            \end{bmatrix}.
        \end{align*}
        Resolvendo o sistema $R y = b$ temos que $y = (16.00, 25.33, 18.67, 18.47)^t$ e resolvendo o sistema $R^t x = y$ temos que $x = (5.11, 16.48, 5.64, 3.69)^t$.
     \end{solution}

    \question Verifique que os elementos do fator de Cholesky s\~{a}o limitados superiormente, isto \'{e}: $| g_{ij} | \leq \alpha$, sendo que $\alpha$ depende apenas das entradas de $A$. Este fato ocorre na fatora\c{c}\~{a}o LU? Justifique e exemplifique.
    \begin{solution}
        Seja
        \begin{align*}
            A &= \begin{bmatrix}
                A_{11} & A_{\mdot 1}^t \\
                A_{\mdot 1} & A_{22}
            \end{bmatrix}
        \end{align*}
        uma matriz sim\'{e}trica definida positiva. Ent\~{a}o a fatora\c{c}\~{a}o de Cholesky corresponde a
        \begin{align*}
            A = G G^ = \begin{bmatrix}
                G_{11} & 0^t \\
                G_{\mdot 1} & G_{22}
            \end{bmatrix} \begin{bmatrix}
                G_{11} & G_{\mdot 1}^t \\
                0 & G_{22}^t
            \end{bmatrix},
        \end{align*}
        onde $G_{11} = \sqrt{A_{11}}$, $G_{\mdot 1} = A_{\mdot 1} / G_{11}$ e $G_{22} G_{22}^t = A_{22} - G_{\mdot 1} G_{\mdot 1}^t$.

        Como $G_{\mdot 1} G_{\mdot 1}^t > 0$ temos que os elementos da fatora\c{c}\~{a}o de Cholesky s\~{a}o limitados superiormente e o limitante depende apenas das entradas de $A$.
    \end{solution}

    \question[Ver se\c{c}\~{a}o 4.2.5, p\'{a}gina 144, do Golub\nocite{Golub:1996:matrix}] Usando a vers\~{a}o que utiliza o produto externo verifique que o n\'{u}mero de opera\c{c}\~{o}es realizads para obter o fator de Cholesky \'{e} da ordem de $n^3 / 3$.
    \begin{solution}
        O fator de Cholesky utilizando o produto externo \'{e} obtido pela seguinte parti\c{c}\~{a}o:
        \begin{align*}
            A = \begin{bmatrix}
                \alpha & v^t \\
                v & B
            \end{bmatrix} = \begin{bmatrix}
                \beta & 0 \\
                v / \beta I
            \end{bmatrix} \begin{bmatrix}
                1 & 0 \\
                0 & B - v v^t / \alpha
            \end{bmatrix} \begin{bmatrix}
                \beta & v^t / \beta \\
                0 & I
            \end{bmatrix}.
        \end{align*}

        O Algoritmo 4.2.2, p\'{a}gina 145, do Golub\nocite{Golub:1996:matrix} corresponde a fatora\c{c}\~{a}o de Cholesky utilizando o produto externo e \'{e} transcrito abaixo:
        \begin{algorithmic}
            \For{$k = 1, \ldots, n$}
                \State $A(k,k) = \sqrt{A(k,k)}$ \Comment{$1$ opera\c{c}\~{a}o}
                \State $A(k + 1:n, k) = A(k + 1:n, k) / A(k,k)$ \Comment{$n - k - 1$ opera\c{c}\~{o}es}
                \For{$j = k + 1:n$}
                    \State $A(j:n, j) = A(j:n, j) - A(j:n, k) A(j, k)$ \Comment{$2(n - j)$ opera\c{c}\~{o}es}
                \EndFor
            \EndFor
        \end{algorithmic}
        Logo, temos que o n\'{u}mero total de opera\c{c}\~{o}es, $\text{op}$, \'{e}
        \begin{align*}
            \text{op} &= \sum_{k = 1}^n \left( 1 + \left( n - k - 1 \right) + \sum_{j = k + 1}^n 2 \left( n - j \right) \right) \\
            &= \sum_{k = 1}^n \left( 1 + \left( n - k - 1 \right) + 2 \sum_{j = k + 1}^n \left( n - j \right) \right) \\
            &= \sum_{k = 1}^n \left( 1 + \left( n - k - 1 \right) + 2 \left[ \left( n - \left( k + 1 \right) + \left( n - n \right) \right) \left( n - \left( k + 1 \right) \right) / 2 \right] \right) \\
            &= \sum_{k = 1}^n \left( 1 + \left( n - k - 1 \right) + \left( n - \left( k + 1 \right) \right) \left( n - \left( k + 1 \right) \right) \right) \\
            &= \sum_{k = 1}^n \left( n - k + n^2 - 2 \left( k + 1 \right) + \left( k + 1 \right)^2 \right) \\
            &= \left( n + n^2 \right) + \sum_{k = 1}^n \left( - k  - 2 \left( k + 1 \right) + \left( k + 1 \right)^2 \right) \\
            &\approx \left( n + n^2 \right) + \sum_{k = 1}^n \left( k + 1 \right)^2 \\
            &= \left( n + n^2 \right) + \left[ \left( n + 1 \right) \left( n + 2 \right) \left( 2n + 3 \right) \right]  / 6 \\
            &\approx n^3 / 3.
        \end{align*}
    \end{solution}

    \question Demonstre que: Se a matriz $A : n \times n$ \'{e} sim\'{e}trica definida positiva, e,
    \begin{align*}
        A = \begin{bmatrix}
            A_{11} & A_{12} \\
            A_{21} & A_{22}
        \end{bmatrix},
    \end{align*}
    ent\~{a}o, $A_{11} : k \times k$ e $A_{22}: (n - k) \times (n - k)$ s\~{a}o sim\'{e}tricas definidas positivas.
    \begin{solution}
        Seja
        \begin{align*}
            A &= \begin{bmatrix}
                A_{11} & A_{12} \\
                A_{21} & A_{22}
            \end{bmatrix}
        \end{align*}
        uma matriz definida positiva, i.e., $x^t A x \geq 0$ para qualquer $x$.
        
        Tomemos $x = (x_1^t, 0)^t$, ent\~{a}o
        \begin{align*}
            x^t A x &= \begin{bmatrix}
                x_1^t & 0
            \end{bmatrix} \begin{bmatrix}
                A_{11} & A_{12} \\
                A_{21} & A_{22}
            \end{bmatrix} \begin{bmatrix}
                x_1 \\
                0
            \end{bmatrix} \\
            &= \begin{bmatrix}
                x_1^t & 0
            \end{bmatrix} \begin{bmatrix}
                A_{11} x_1 \\
                A_{21} x_1
            \end{bmatrix} \\
            &= x_1^t A_{11} x_1.
        \end{align*}
        Como $x^t A x \geq 0$ para qualquer $x$ temos que $x_1^t A_{11} x_1 \geq 0$ para qualquer $x_1$ e portanto as submatrizes $A_{11}$ de ordem $k$ s\~{a}o matrizes sim\'{e}tricas definidas positivas.

        Agora tememos $x = (0, x_2^t)^t$, ent\~{a}o
        \begin{align*}
            x^t A x &= \begin{bmatrix}
                0 & x_2^t
            \end{bmatrix} \begin{bmatrix}
                A_{11} & A_{12} \\
                A_{21} & A_{22}
            \end{bmatrix} \begin{bmatrix}
                0 \\
                x_2
            \end{bmatrix} \\
            &= \begin{bmatrix}
                0 & x_2^t
            \end{bmatrix} \begin{bmatrix}
                A_{12} x_2 \\
                A_{22} x_2
            \end{bmatrix} \\
            &= x_2^t A_{22} x_2.
        \end{align*}
        Como $x^t A x \geq 0$ para qualquer $x$ temos que $x_2^t A_{22} x_2 \geq 0$ para qualque $x_2$ e portanto as submatrizes $A_{22}$ de ordem $(n - k)$ s\~{a}o matrizes sim\'{e}tricas definidas positivas
    \end{solution}
\end{questions}
\bibliographystyle{plain}
\bibliography{bibliography}
\end{document}
